\chapter{Introduction}
\label{chapter1}

MapReduce~\cite{Dean_Ghemawat_OSDI2004} is a programming model for
expressing distributed computations on massive amounts of data and an
execution framework for large-scale data processing on clusters of
commodity servers.  It was originally developed by Google and built on
well-known principles in parallel and distributed processing dating
back several decades.  MapReduce has since enjoyed widespread adoption
via an open-source implementation called Hadoop, whose development was
led by Yahoo (now an Apache project).  Today, a vibrant software
ecosystem has sprung up around Hadoop, with significant activity in
both industry and academia.

This book is about scalable approaches to processing large amounts of
text with MapReduce.  Given this focus, it makes sense to start with
the most basic question:\ Why?  There are many answers to this
question, but we focus on two.  First, ``big data'' is a fact of the
world, and therefore an issue that real-world systems must grapple
with.  Second, across a wide range of text processing applications,
more data translates into more effective algorithms, and thus it makes
sense to take advantage of the plentiful amounts of data that surround
us.

Modern information societies are defined by vast repositories of data,
both public and private.  Therefore, any practical application must be
able to scale up to datasets of interest.  For many, this means
scaling up to the web, or at least a non-trivial fraction thereof.
Any organization built around gathering, analyzing, monitoring,
filtering, searching, or organizing web content must tackle large-data
problems:\ ``web-scale'' processing is practically synonymous with
data-intensive processing.  This observation applies not only to
well-established internet companies, but also countless startups and
niche players as well.  Just think, how many companies do you know
that start their pitch with ``we're going to harvest information on
the web and\ldots''?

Another strong area of growth is the analysis of user behavior data.
Any operator of a moderately successful website can record user
activity and in a matter of weeks (or sooner) be drowning in a torrent
of log data.  In fact, logging user behavior generates so much data
that many organizations simply can't cope with the volume, and either
turn the functionality off or throw away data after some time.  This
represents lost opportunities, as there is a broadly-held belief that
great value lies in insights derived from mining such data.  Knowing
what users look at, what they click on, how much time they spend on a
web page, etc.\ leads to better business decisions and competitive
advantages.  Broadly, this is known as business intelligence, which
encompasses a wide range of technologies including data warehousing,
data mining, and analytics.

How much data are we talking about?  A few examples: Google grew from
processing 100 TB of data a day with MapReduce in
2004~\cite{Dean_Ghemawat_OSDI2004} to processing 20 PB a day with
MapReduce in 2008~\cite{Dean_Ghemawat_CACM2008}.  In April 2009, a
blog post\footnote{\tt
  http://www.dbms2.com/2009/04/30/ebays-two-enormous-data-warehouses/}
was written about eBay's two enormous data warehouses:\ one with 2
petabytes of user data, and the other with 6.5 petabytes of user data
spanning 170 trillion records and growing by 150 billion new records
per day.  Shortly thereafter, Facebook revealed\footnote{\tt
  http://www.dbms2.com/2009/05/11/facebook-hadoop-and-hive/} similarly
impressive numbers, boasting of 2.5 petabytes of user data, growing at
about 15 terabytes per day.  Petabyte datasets are rapidly becoming
the norm, and the trends are clear:\ our ability to store data is fast
overwhelming our ability to process what we store.  More distressing,
increases in capacity are outpacing improvements in bandwidth such
that our ability to even {\it read} back what we store is
deteriorating~\cite{Leventhal_2009}.  Disk capacities have grown from
tens of megabytes in the mid-1980s to about a couple of terabytes
today (several orders of magnitude).  On the other hand, latency and
bandwidth have improved relatively little:\ in the case of latency,
perhaps 2$\times$ improvement during the last quarter century, and in
the case of bandwidth, perhaps 50$\times$.  Given the tendency for
individuals and organizations to continuously fill up whatever
capacity is available, large-data problems are growing increasingly
severe.

Moving beyond the commercial sphere, many have recognized the
importance of data management in many scientific disciplines, where
petabyte-scale datasets are also becoming increasingly
common~\cite{Bell_etal_2009}.  For example:

\begin{itemize}

\item The high-energy physics community was already describing
  experiences with petabyte-scale databases back in
  2005~\cite{Becla_Wang_2005}.  Today, the Large Hadron Collider (LHC)
  near Geneva is the world's largest particle accelerator, designed to
  probe the mysteries of the universe, including the fundamental
  nature of matter, by recreating conditions shortly following the Big
  Bang.  When it becomes fully operational, the LHC will produce
  roughly 15 petabytes of data a year.\footnote{\tt
    http://public.web.cern.ch/public/en/LHC/Computing-en.html}

\item Astronomers have long recognized the importance of a ``digital
  observatory'' that would support the data needs of researchers
  across the globe---the Sloan Digital Sky
  Survey~\cite{Szalay_etal_2000} is perhaps the most well known of
  these projects.  Looking into the future, the Large Synoptic Survey
  Telescope (LSST) is a wide-field instrument that is capable of
  observing the entire sky every few days.  When the telescope comes
  online around 2015 in Chile, its 3.2 gigapixel primary camera will
  produce approximately half a petabyte of archive images every
  month~\cite{Becla_etal_2006}.

\item The advent of next-generation DNA sequencing technology has
  created a deluge of sequence data that needs to be stored,
  organized, and delivered to scientists for further study.  Given the
  fundamental tenant in modern genetics that genotypes explain
  phenotypes, the impact of this technology is nothing less than
  transformative~\cite{Mardis_2008}.  The European Bioinformatics
  Institute (EBI), which hosts a central repository of sequence data
  called EMBL-bank, has increased storage capacity from 2.5 petabytes
  in 2008 to 5 petabytes in 2009~\cite{Southan_Cameron_2009}.
  Scientists are predicting that, in the not-so-distant future,
  sequencing an individual's genome will be no more complex than
  getting a blood test today---ushering a new era of personalized
  medicine, where interventions can be specifically targeted for an
  individual.

\end{itemize}

\noindent Increasingly, scientific breakthroughs will be powered by
advanced computing capabilities that help researchers manipulate,
explore, and mine massive datasets~\cite{Hey_etal_2009}---this has
been hailed as the emerging ``fourth paradigm'' of
science~\cite{Hey_etal_2009-Gray} (complementing theory, experiments,
and simulations).  In other areas of academia, particularly computer
science, systems and algorithms incapable of scaling to massive
real-world datasets run the danger of being dismissed as ``toy
systems'' with limited utility.  Large data is a fact of today's world
and data-intensive processing is fast becoming a necessity, not merely
a luxury or curiosity.

Although large data comes in a variety of forms, this book is
primarily concerned with processing large amounts of text, but touches
on other types of data as well (e.g., relational and graph data). The
problems and solutions we discuss mostly fall into the disciplinary
boundaries of natural language processing (NLP) and information
retrieval (IR).  Recent work in these fields is dominated by a
data-driven, empirical approach, typically involving algorithms that
attempt to capture statistical regularities in data for the purposes
of some task or application.  There are three components to this
approach:\ data, representations of the data, and some method for
capturing regularities in the data.  Data are called {\it corpora}
(singular, corpus) by NLP researchers and {\it collections} by those
from the IR community.  Aspects of the representations of the data are
called {\it features}, which may be ``superficial'' and easy to
extract, such as the words and sequences of words themselves, or
``deep'' and more difficult to extract, such as the grammatical
relationship between words.  Finally, algorithms or models are applied
to capture regularities in the data in terms of the extracted features
for some application.  One common application, classification, is to
sort text into categories.  Examples include:\ Is this email spam or
not spam?  Is this word part of an address or a location?  The first
task is easy to understand, while the second task is an instance of
what NLP researchers call named-entity
detection~\cite{Sekine_Ranchhod_2009}, which is useful for local
search and pinpointing locations on maps.  Another common application
is to rank texts according to some criteria---search is a good
example, which involves ranking documents by relevance to the user's
query.  Another example is to automatically situate texts along a
scale of ``happiness'', a task known as sentiment analysis or opinion
mining~\cite{Pang_Lee_2008}, which has been applied to everything from
understanding political discourse in the blogosphere to predicting the
movement of stock prices.

There is a growing body of evidence, at least in text processing, that
of the three components discussed above (data, features, algorithms),
data probably matters the most.  Superficial word-level features
coupled with simple models in most cases trump sophisticated models
over deeper features and less data.  But why can't we have our cake
and eat it too?  Why not both sophisticated models {\it and} deep
features applied to lots of data?  Because inference over
sophisticated models and extraction of deep features are often
computationally intensive, they don't scale well.

Consider a simple task such as determining the correct usage of easily
confusable words such as ``than'' and ``then'' in English.  One can
view this as a supervised machine learning problem:\ we can train a
classifier to disambiguate between the options, and then apply the
classifier to new instances of the problem (say, as part of a grammar
checker).  Training data is fairly easy to come by---we can just
gather a large corpus of texts and assume that most writers make
correct choices (the training data may be noisy, since people make
mistakes, but no matter).  In 2001, Banko and Brill~\cite{Banko01}
published what has become a classic paper in natural language
processing exploring the effects of training data size on
classification accuracy, using this task as the specific example.
They explored several classification algorithms (the exact ones aren't
important, as we shall see), and not surprisingly, found that more
data led to better accuracy.  Across many different algorithms, the
increase in accuracy was approximately linear in the log of the size
of the training data.  Furthermore, with increasing amounts of
training data, the accuracy of different algorithms converged, such
that pronounced differences in effectiveness observed on smaller
datasets basically disappeared at scale.  This led to a somewhat
controversial conclusion (at least at the time):\ machine learning
algorithms really don't matter, all that matters is the amount of data
you have.  This resulted in an even more controversial recommendation,
delivered somewhat tongue-in-cheek:\ we should just give up working on
algorithms and simply spend our time gathering data (while waiting for
computers to become faster so we can process the data).

As another example, consider the problem of answering short,
fact-based questions such as ``Who shot Abraham Lincoln?''  Instead of
returning a list of documents that the user would then have to sort
through, a question answering (QA) system would directly return the
answer:\ John Wilkes Booth.  This problem gained interest in the late
1990s, when natural language processing researchers approached the
challenge with sophisticated linguistic processing techniques such as
syntactic and semantic analysis.  Around 2001, researchers discovered
a far simpler approach to answering such questions based on pattern
matching~\cite{Brill_etal_TREC2001,Dumais_etal_SIGIR2002,Lin_TOIS2007}.
Suppose you wanted the answer to the above question.  As it turns out,
you can simply search for the phrase ``shot Abraham Lincoln'' on the
web and look for what appears to its left.  Or better yet, look
through multiple instances of this phrase and tally up the words that
appear to the left.  This simple strategy works surprisingly well, and
has become known as the {\it redundancy-based approach} to question
answering.  It capitalizes on the insight that in a very large text
collection (i.e., the web), answers to commonly-asked questions will
be stated in obvious ways, such that pattern-matching techniques
suffice to extract answers accurately.

Yet another example concerns smoothing in web-scale language
models~\cite{Brants_etal_EMNLP2007}.  A language model is a
probability distribution that characterizes the likelihood of
observing a particular sequence of words, estimated from a large
corpus of texts.  They are useful in a variety of applications, such
as speech recognition (to determine what the speaker is more likely to
have said) and machine translation (to determine which of possible
translations is the most fluent, as we will discuss in
Section~\ref{chapter6_word_alignment}).  Since there are infinitely
many possible strings, and probabilities must be assigned to all of
them, language modeling is a more challenging task than simply keeping
track of which strings were seen how many times:\ some number of
likely strings will never be encountered, even with lots and lots of
training data!  Most modern language models make the Markov
assumption:\ in a {\it n}-gram language model, the conditional
probability of a word is given by the $n-1$ previous words.  Thus, by
the chain rule, the probability of a sequence of words can be
decomposed into the product of {\it n}-gram probabilities.
Nevertheless, an enormous number of parameters must still be estimated
from a training corpus:\ potentially $V^n$ parameters, where $V$ is
the number of words in the vocabulary.  Even if we treat every word on
the web as the training corpus from which to estimate the $n$-gram
probabilities, most $n$-grams---in any language, even English---will
never have been seen.  To cope with this sparseness, researchers have
developed a number of smoothing
techniques~\cite{Chen_Goodman_ACL1996,Manning_Schutze_1999,Jurafsky_Martin_2009},
which all share the basic idea of moving probability mass from
observed to unseen events in a principled manner.  Smoothing
approaches vary in effectiveness, both in terms of intrinsic and
application-specific metrics.  In 2007, Brants et
al.~\cite{Brants_etal_EMNLP2007} described language models trained on
up to two trillion words.\footnote{As an aside, it is interesting to
  observe the evolving definition of {\it large} over the years.
  Banko and Brill's paper in 2001 was titled {\it Scaling to Very Very
    Large Corpora for Natural Language Disambiguation}, and dealt with
  a corpus containing a billion words.}  Their experiments compared a
state-of-the-art approach known as Kneser-Ney
smoothing~\cite{Chen_Goodman_ACL1996} with another technique the
authors affectionately referred to as ``stupid backoff''.\footnote{As
  in, so stupid it couldn't possibly work.}  Not surprisingly, stupid
backoff didn't work as well as Kneser-Ney smoothing on smaller
corpora.  However, it was simpler and could be trained on {\it more}
data, which ultimately yielded better language models.  That is, a
simpler technique on more data beat a more sophisticated technique on
less data.

Recently, three Google researchers summarized this data-driven
philosophy in an essay titled {\it The Unreasonable Effectiveness of
  Data}~\cite{Halevy_etal_2009}.\footnote{This title was inspired by a
  classic article titled {\it The Unreasonable Effectiveness of
    Mathematics in the Natural Sciences}~\cite{Wigner_1960}.  This is
  somewhat ironic in that the original article lauded the beauty and
  elegance of mathematical models in capturing natural phenomena,
  which is the exact opposite of the data-driven approach.} Why is
this so?  It boils down to the fact that language {\it in the wild},
just like human behavior in general, is messy.  Unlike, say, the
interaction of subatomic particles, human {\it use} of language is not
constrained by succinct, universal ``laws of grammar''.  There are of
course rules that govern the formation of words and sentences---for
example, that verbs appear before objects in English, and that
subjects and verbs must agree in number in many languages---but
real-world language is affected by a multitude of other factors as
well:\ people invent new words and phrases all the time, authors
occasionally make mistakes, groups of individuals write within a
shared context, etc.  The Argentine writer Jorge Luis Borges wrote a
famous allegorical one-paragraph story about a fictional society in
which the art of cartography had gotten so advanced that their maps
were as big as the lands they were describing.\footnote{{\it On
    Exactitude in Science}~\cite{Borges_1999}.  A similar exchange
  appears in Chapter XI of {\it Sylvie and Bruno Concluded} by Lewis
  Carroll (1893).} The world, he would say, is the best description of
itself.  In the same way, the more observations we gather about
language use, the more accurate a description we have of language
itself.  This, in turn, translates into more effective algorithms and
systems.

So, in summary, why large data?  In some ways, the first answer is
similar to the reason people climb mountains:\ because they're there.
But the second answer is even more compelling.  Data represent the
rising tide that lifts all boats---more data lead to better algorithms
and systems for solving real-world problems.  Now that we've addressed
the {\it why}, let's tackle the {\it how}.  Let's start with the
obvious observation:\ data-intensive processing is beyond the
capability of any individual machine and requires clusters---which
means that large-data problems are fundamentally about organizing
computations on dozens, hundreds, or even thousands of machines.  This
is exactly what MapReduce does, and the rest of this book is about the
{\it how}.

\section{Computing in the Clouds}
\label{chapter1:clouds}

For better or for worse, it is often difficult to untangle MapReduce
and large-data processing from the broader discourse on cloud
computing.  True, there is substantial promise in this new paradigm of
computing, but unwarranted hype by the media and popular sources
threatens its credibility in the long run.  In some ways, cloud
computing is simply brilliant marketing.  Before clouds, there were
grids,\footnote{What {\it is} the difference between cloud computing
  and grid computing?  Although both tackle the fundamental problem of
  how best to bring computational resources to bear on large and
  difficult problems, they start with different assumptions.  Whereas
  clouds are assumed to be relatively homogeneous servers that reside
  in a datacenter or are distributed across a relatively small number
  of datacenters controlled by a single organization, grids are
  assumed to be a less tightly-coupled federation of heterogeneous
  resources under the control of distinct but cooperative
  organizations.  As a result, grid computing tends to deal with tasks
  that are coarser-grained, and must deal with the practicalities of a
  federated environment, e.g., verifying credentials across multiple
  administrative domains.  Grid computing has adopted a
  middleware-based approach for tackling many of these challenges.}
and before grids, there were vector supercomputers, each having
claimed to be the best thing since sliced bread.

So what exactly is cloud computing?  This is one of those questions
where ten experts will give eleven different answers; in fact,
countless papers have been written simply to attempt to define the
term
(e.g.,~\cite{Armbrust_etal_2009,Buyya_etal_2009,Vaquero_etal_2009},
just to name a few examples).  Here we offer up our own thoughts and
attempt to explain how cloud computing relates to MapReduce and
data-intensive processing.

At the most superficial level, everything that used to be called web
applications has been rebranded to become ``cloud applications'',
which includes what we have previously called ``Web 2.0'' sites.  In
fact, anything running inside a browser that gathers and stores
user-generated content now qualifies as an example of cloud computing.
This includes social-networking services such as Facebook,
video-sharing sites such as YouTube, web-based email services such as
Gmail, and applications such as Google Docs.  In this context, the
cloud simply refers to the servers that power these sites, and user
data is said to reside ``in the cloud''.  The accumulation of vast
quantities of user data creates large-data problems, many of which are
suitable for MapReduce.  To give two concrete examples:\ a
social-networking site analyzes connections in the enormous
globe-spanning graph of friendships to recommend new connections.  An
online email service analyzes messages and user behavior to optimize
ad selection and placement.  These are all large-data problems that
have been tackled with MapReduce.\footnote{The first example is
  Facebook, a well-known user of Hadoop, in exactly the manner as
  described~\cite{Hammerbacher_2009}.  The second is, of course,
  Google, which uses MapReduce to continuously improve existing
  algorithms and to devise new algorithms for ad selection and
  placement.}

Another important facet of cloud computing is what's more precisely
known as utility computing~\cite{Rappa_2004,Buyya_etal_2009}.  As the
name implies, the idea behind utility computing is to treat computing
resource as a metered service, like electricity or natural gas.  The
idea harkens back to the days of time-sharing machines, and in truth
isn't very different from this antiquated form of computing.  Under
this model, a ``cloud user'' can dynamically provision any amount of
computing resources from a ``cloud provider'' on demand and only pay
for what is consumed.  In practical terms, the user is paying for
access to virtual machine instances that run a standard operating
system such as Linux.  Virtualization technology
(e.g.,~\cite{Barham_etal_2003}) is used by the cloud provider to
allocate available physical resources and enforce isolation between
multiple users that may be sharing the same hardware.  Once one or
more virtual machine instances have been provisioned, the user has
full control over the resources and can use them for arbitrary
computation.  Virtual machines that are no longer needed are
destroyed, thereby freeing up physical resources that can be
redirected to other users.  Resource consumption is measured in some
equivalent of machine-hours and users are charged in increments
thereof.

Both users and providers benefit in the utility computing model.
Users are freed from upfront capital investments necessary to build
datacenters and substantial reoccurring costs in maintaining them.
They also gain the important property of elasticity---as demand for
computing resources grow, for example, from an unpredicted spike in
customers, more resources can be seamlessly allocated from the cloud
without an interruption in service.  As demand falls, provisioned
resources can be released.  Prior to the advent of utility computing,
coping with unexpected spikes in demand was fraught with
challenges:\ under-provision and run the risk of service
interruptions, or over-provision and tie up precious capital in idle
machines that are depreciating.

From the utility provider point of view, this business also makes
sense because large datacenters benefit from economies of scale and
can be run more efficiently than smaller datacenters.  In the same way
that insurance works by aggregating risk and redistributing it,
utility providers aggregate the computing demands for a large number
of users.  Although demand may fluctuate significantly for each user,
overall trends in aggregate demand should be smooth and predictable,
which allows the cloud provider to adjust capacity over time with less
risk of either offering too much (resulting in inefficient use of
capital) or too little (resulting in unsatisfied customers).  In the
world of utility computing, Amazon Web Services currently leads the
way and remains the dominant player, but a number of other cloud
providers populate a market that is becoming increasingly crowded.
Most systems are based on proprietary infrastructure, but there is at
least one, Eucalyptus~\cite{Nurmi_etal_2009}, that is available open
source.  Increased competition will benefit cloud users, but what
direct relevance does this have for MapReduce?  The connection is
quite simple:\ processing large amounts of data with MapReduce
requires access to clusters with sufficient capacity.  However, not
everyone with large-data problems can afford to purchase and maintain
clusters.  This is where utility computing comes in:\ clusters of
sufficient size can be provisioned only when the need arises, and
users pay only as much as is required to solve their problems.  This
lowers the barrier to entry for data-intensive processing and makes
MapReduce much more accessible.

A generalization of the utility computing concept is ``everything as a
service'', which is itself a new take on the age-old idea of
outsourcing.  A cloud provider offering customers access to virtual
machine instances is said to be offering infrastructure as a service,
or IaaS for short.  However, this may be too low level for many users.
Enter platform as a service (PaaS), which is a rebranding of what used
to be called hosted services in the ``pre-cloud'' era.  Platform is
used generically to refer to any set of well-defined services on top
of which users can build applications, deploy content, etc.  This
class of services is best exemplified by Google App Engine, which
provides the backend datastore and API for anyone to build
highly-scalable web applications.  Google maintains the
infrastructure, freeing the user from having to backup, upgrade,
patch, or otherwise maintain basic services such as the storage layer
or the programming environment.  At an even higher level, cloud
providers can offer software as a service (SaaS), as exemplified by
Salesforce, a leader in customer relationship management (CRM)
software.  Other examples include outsourcing an entire organization's
email to a third party, which is commonplace today.

What does this proliferation of services have to do with MapReduce?
No doubt that ``everything as a service'' is driven by desires for
greater business efficiencies, but scale and elasticity play important
roles as well.  The cloud allows seamless expansion of operations
without the need for careful planning and supports scales that may
otherwise be difficult or cost-prohibitive for an organization to
achieve.  Cloud services, just like MapReduce, represents the search
for an appropriate level of abstraction and beneficial divisions of
labor.  IaaS is an abstraction over raw physical hardware---an
organization might lack the capital, expertise, or interest in running
datacenters, and therefore pays a cloud provider to do so on its
behalf.  The argument applies similarly to PaaS and SaaS.  In the same
vein, the MapReduce programming model is a powerful abstraction that
separates the {\it what} from the {\it how} of data-intensive
processing.

\section{Big Ideas}

Tackling large-data problems requires a distinct approach that
sometimes runs counter to traditional models of computing.  In this
section, we discuss a number of ``big ideas'' behind MapReduce.  To be
fair, all of these ideas have been discussed in the computer science
literature for some time (some for decades), and MapReduce is
certainly not the first to adopt these ideas.  Nevertheless, the
engineers at Google deserve tremendous credit for pulling these
various threads together and demonstrating the power of these ideas on
a scale previously unheard of.

\paragraph{Scale ``out'', not ``up''.}
For data-intensive workloads, a large number of commodity low-end
servers (i.e., the scaling ``out'' approach) is preferred over a small
number of high-end servers (i.e., the scaling ``up'' approach).  The
latter approach of purchasing symmetric multi-processing (SMP)
machines with a large number of processor sockets (dozens, even
hundreds) and a large amount of shared memory (hundreds or even
thousands of gigabytes) is not cost effective, since the costs of such
machines do not scale linearly (i.e., a machine with twice as many
processors is often significantly more than twice as expensive).  On
the other hand, the low-end server market overlaps with the
high-volume desktop computing market, which has the effect of keeping
prices low due to competition, interchangeable components, and
economies of scale.

Barroso and H\"{o}lzle's recent treatise of what they dubbed
``warehouse-scale computers''~\cite{Barroso_Holzle_2009} contains a
thoughtful analysis of the two approaches.  The Transaction Processing
Council (TPC) is a neutral, non-profit organization whose mission is
to establish objective database benchmarks.  Benchmark data submitted
to that organization are probably the closest one can get to a fair
``apples-to-apples'' comparison of cost and performance for specific,
well-defined relational processing applications.  Based on
\mbox{TPC-C} benchmark results from late 2007, a low-end server
platform is about four times more cost efficient than a high-end
shared memory platform from the same vendor.  Excluding storage costs,
the price/performance advantage of the low-end server increases to
about a factor of twelve.

What if we take into account the fact that communication between nodes
in a high-end SMP machine is orders of magnitude faster than
communication between nodes in a commodity network-based cluster?
Since workloads today are beyond the capability of any {\it single}
machine (no matter how powerful), the comparison is more accurately
between a smaller cluster of high-end machines and a larger cluster of
low-end machines (network communication is unavoidable in both cases).
Barroso and H\"{o}lzle model these two approaches under workloads that
demand more or less communication, and conclude that a cluster of
low-end servers approaches the performance of the equivalent cluster
of high-end servers---the small performance gap is insufficient to
justify the price premium of the high-end servers.  For data-intensive
applications, the conclusion appears to be clear:\ scaling ``out'' is
superior to scaling ``up'', and therefore most existing
implementations of the MapReduce programming model are designed around
clusters of low-end commodity servers.

Capital costs in acquiring servers is, of course, only one component
of the total cost of delivering computing capacity.  Operational costs
are dominated by the cost of electricity to power the servers as well
as other aspects of datacenter operations that are functionally
related to power:\ power distribution, cooling,
etc.~\cite{Hamilton_2009,Barroso_Holzle_2009}.  As a result, energy
efficiency has become a key issue in building warehouse-scale
computers for large-data processing.  Therefore, it is important to
factor in operational costs when deploying a scale-out solution based
on large numbers of commodity servers.

Datacenter efficiency is typically factored into three separate
components that can be independently measured and
optimized~\cite{Barroso_Holzle_2009}.  The first component measures
how much of a building's incoming power is actually delivered to
computing equipment, and correspondingly, how much is lost to the
building's mechanical systems (e.g., cooling, air handling) and
electrical infrastructure (e.g., power distribution inefficiencies).
The second component measures how much of a server's incoming power is
lost to the power supply, cooling fans, etc.  The third component
captures how much of the power delivered to computing components
(processor, RAM, disk, etc.) is actually used to perform useful
computations.

Of the three components of datacenter efficiency, the first two are
relatively straightforward to objectively quantify.  Adoption of
industry best-practices can help datacenter operators achieve
state-of-the-art efficiency.  The third component, however, is much
more difficult to measure.  One important issue that has been
identified is the non-linearity between load and power draw.  That is,
a server at 10\% utilization may draw slightly more than half as much
power as a server at 100\% utilization (which means that a
lightly-loaded server is much less efficient than a heavily-loaded
server).  A survey of five thousand Google servers over a six-month
period shows that servers operate most of the time at between 10\% and
50\% utilization~\cite{Barroso_Holzle_2007}, which is an
energy-inefficient operating region.  As a result, Barroso and
H\"{o}lzle have advocated for research and development in
energy-proportional machines, where energy consumption would be
proportional to load, such that an idle processor would (ideally)
consume no power, but yet retain the ability to power up (nearly)
instantaneously in response to demand.

Although we have provided a brief overview here, datacenter efficiency
is a topic that is beyond the scope of this book.  For more details,
consult Barroso and H\"{o}lzle~\cite{Barroso_Holzle_2009} and
Hamilton~\cite{Hamilton_2009}, who provide detailed cost models for
typical modern datacenters.  However, even factoring in operational
costs, evidence suggests that scaling out remains more attractive than
scaling up.

\paragraph{Assume failures are common.} 
At warehouse scale, failures are not only inevitable, but commonplace.
A simple calculation suffices to demonstrate:\ let us suppose that a
cluster is built from reliable machines with a mean-time between
failures (MTBF) of 1000 days (about three years).  Even with these
reliable servers, a 10,000-server cluster would still experience
roughly 10 failures a day.  For the sake of argument, let us suppose
that a MTBF of 10,000 days (about thirty years) were achievable at
realistic costs (which is unlikely).  Even then, a 10,000-server
cluster would still experience one failure daily.  This means that any
large-scale service that is distributed across a large cluster (either
a user-facing application or a computing platform like MapReduce) must
cope with hardware failures as an intrinsic aspect of its
operation~\cite{Hamilton_2007}.  That is, a server may fail at any
time, without notice.  For example, in large clusters disk failures
are common~\cite{Pinheiro_etal_2007} and RAM experiences more errors
than one might expect~\cite{Schroeder_etal_2009}.  Datacenters suffer
from both planned outages (e.g., system maintenance and hardware
upgrades) and unexpected outages (e.g., power failure, connectivity
loss, etc.).

A well-designed, fault-tolerant service must cope with failures up to
a point without impacting the quality of service---failures should not
result in inconsistencies or indeterminism from the user perspective.
As servers go down, other cluster nodes should seamlessly step in to
handle the load, and overall performance should gracefully degrade as
server failures pile up.  Just as important, a broken server that has
been repaired should be able to seamlessly rejoin the service without
manual reconfiguration by the administrator.  Mature implementations of
the MapReduce programming model are able to robustly cope with
failures through a number of mechanisms such as automatic task
restarts on different cluster nodes.

\paragraph{Move processing to the data.}
In traditional high-performance computing (HPC) applications (e.g.,
for climate or nuclear simulations), it is commonplace for a
supercomputer to have ``processing nodes'' and ``storage nodes''
linked together by a high-capacity interconnect.  Many data-intensive
workloads are not very processor-demanding, which means that the
separation of compute and storage creates a bottleneck in the network.
As an alternative to moving data around, it is more efficient to move
the processing around.  That is, MapReduce assumes an architecture
where processors and storage (disk) are co-located.  In such a setup,
we can take advantage of data locality by running code on the
processor directly attached to the block of data we need.  The
distributed file system is responsible for managing the data over
which MapReduce operates.

\paragraph{Process data sequentially and avoid random access.}
Data-intensive processing by definition means that the relevant
datasets are too large to fit in memory and must be held on disk.
Seek times for random disk access are fundamentally limited by the
mechanical nature of the devices:\ read heads can only move so fast
and platters can only spin so rapidly.  As a result, it is desirable
to avoid random data access, and instead organize computations so that
data is processed sequentially.  A simple scenario\footnote{Adapted
  from a post by Ted Dunning on the Hadoop mailing list.} poignantly
illustrates the large performance gap between sequential operations
and random seeks:\ assume a 1 terabyte database containing 10$^{10}$
100-byte records.  Given reasonable assumptions about disk latency and
throughput, a back-of-the-envelop calculation will show that updating
1$\%$ of the records (by accessing and then mutating each record) will
take about a month on a single machine.  On the other hand, if one
simply reads the entire database and rewrites all the records
(mutating those that need updating), the process would finish in under
a work day on a single machine.  Sequential data access is, literally,
orders of magnitude faster than random data access.\footnote{For more
  detail, Jacobs~\cite{JacobsAdam_2009} provides real-world benchmarks
  in his discussion of large-data problems.}

The development of solid-state drives is unlikely the change this
balance for at least two reasons.  First, the cost differential
between traditional magnetic disks and solid-state disks remains
substantial:\ large-data will for the most part remain on mechanical
drives, at least in the near future.  Second, although solid-state
disks have substantially faster seek times, order-of-magnitude
differences in performance between sequential and random access still
remain.

MapReduce is primarily designed for batch processing over large
datasets.  To the extent possible, all computations are organized into
long streaming operations that take advantage of the aggregate
bandwidth of many disks in a cluster.  Many aspects of MapReduce's
design explicitly trade latency for throughput.

\paragraph{Hide system-level details from the application developer.}  
According to many guides on the practice of software engineering
written by experienced industry professionals, one of the key reasons
why writing code is difficult is because the programmer must
simultaneously keep track of many details in short term
memory---ranging from the mundane (e.g., variable names) to the
sophisticated (e.g., a corner case of an algorithm that requires
special treatment).  This imposes a high cognitive load and requires
intense concentration, which leads to a number of recommendations
about a programmer's environment (e.g., quiet office, comfortable
furniture, large monitors, etc.).  The challenges in writing
distributed software are greatly compounded---the programmer must
manage details across several threads, processes, or machines.  Of
course, the biggest headache in distributed programming is that code
runs concurrently in unpredictable orders, accessing data in
unpredictable patterns.  This gives rise to race conditions,
deadlocks, and other well-known problems.  Programmers are taught to
use low-level devices such as mutexes and to apply high-level ``design
patterns'' such as producer--consumer queues to tackle these
challenges, but the truth remains:\ concurrent programs are
notoriously difficult to reason about and even harder to debug.

MapReduce addresses the challenges of distributed programming by
providing an abstraction that isolates the developer from system-level
details (e.g., locking of data structures, data starvation issues in
the processing pipeline, etc.).  The programming model specifies
simple and well-defined interfaces between a small number of
components, and therefore is easy for the programmer to reason about.
MapReduce maintains a separation of {\it what} computations are to be
performed and {\it how} those computations are actually carried out on
a cluster of machines.  The first is under the control of the
programmer, while the second is exclusively the responsibility of the
execution framework or ``runtime''.  The advantage is that the
execution framework only needs to be designed once and verified for
correctness---thereafter, as long as the developer expresses
computations in the programming model, code is guaranteed to behave as
expected.  The upshot is that the developer is freed from having to
worry about system-level details (e.g., no more debugging race
conditions and addressing lock contention) and can instead focus on
algorithm or application design.

\paragraph{Seamless scalability.}
For data-intensive processing, it goes without saying that scalable
algorithms are highly desirable.  As an aspiration, let us sketch the
behavior of an ideal algorithm.  We can define scalability along at
least two dimensions.\footnote{See also DeWitt and
  Gray~\cite{DeWitt_Gray_CACM1992} for slightly different definitions
  in terms of {\it speedup} and {\it scaleup}.}  First, in terms of
data:\ given twice the amount of data, the same algorithm should take
at most twice as long to run, all else being equal.  Second, in terms
of resources:\ given a cluster twice the size, the same algorithm
should take no more than half as long to run.  Furthermore, an ideal
algorithm would maintain these desirable scaling characteristics
across a wide range of settings:\ on data ranging from gigabytes to
petabytes, on clusters consisting of a few to a few thousand machines.
Finally, the ideal algorithm would exhibit these desired behaviors
without requiring any modifications whatsoever, not even tuning of
parameters.

Other than for embarrassingly parallel problems, algorithms with the
characteristics sketched above are, of course, unobtainable.  One of
the fundamental assertions in Fred Brook's classic {\it The Mythical
  Man-Month}~\cite{Brooks_1995} is that adding programmers to a
project behind schedule will only make it fall further behind.  This
is because complex tasks cannot be chopped into smaller pieces and
allocated in a linear fashion, and is often illustrated with a cute
quote:\ ``nine women cannot have a baby in one month''.  Although
Brook's observations are primarily about software engineers and the
software development process, the same is also true of
algorithms:\ increasing the degree of parallelization also increases
communication costs.  The algorithm designer is faced with diminishing
returns, and beyond a certain point, greater efficiencies gained by
parallelization are entirely offset by increased communication
requirements.

Nevertheless, these fundamental limitations shouldn't prevent us from
at least striving for the unobtainable.  The truth is that most
current algorithms are far from the ideal.  In the domain of text
processing, for example, most algorithms today assume that data fits
in memory on a single machine.  For the most part, this is a fair
assumption.  But what happens when the amount of data doubles in the
near future, and then doubles again shortly thereafter?  Simply buying
more memory is not a viable solution, as the amount of data is growing
faster than the price of memory is falling.  Furthermore, the price of
a machine does not scale linearly with the amount of available memory
beyond a certain point (once again, the scaling ``up'' vs.\ scaling
``out'' argument).  Quite simply, algorithms that require holding
intermediate data in memory on a single machine will simply break on
sufficiently-large datasets---moving from a single machine to a
cluster architecture requires fundamentally different algorithms (and
reimplementations).

Perhaps the most exciting aspect of MapReduce is that it represents a
small step toward algorithms that behave in the ideal manner discussed
above.  Recall that the programming model maintains a clear separation
between {\it what} computations need to occur with {\it how} those
computations are actually orchestrated on a cluster.  As a result, a
MapReduce algorithm remains fixed, and it is the responsibility of the
execution framework to execute the algorithm.  Amazingly, the
MapReduce programming model is simple enough that it is actually
possible, in many circumstances, to {\it approach} the ideal scaling
characteristics discussed above.  We introduce the idea of the
``tradeable machine hour'', as a play on Brook's classic title.  If
running an algorithm on a particular dataset takes 100 machine hours,
then we should be able to finish in an hour on a cluster of 100
machines, or use a cluster of 10 machines to complete the same task in
ten hours.\footnote{Note that this idea meshes well with utility
  computing, where a 100-machine cluster running for one hour would
  cost the same as a 10-machine cluster running for ten hours.} With
MapReduce, this isn't so far from the truth, at least for some
applications.

\section{Why Is This Different?}

\begin{quote}
``Due to the rapidly decreasing cost of processing, memory, and
  communication, it has appeared inevitable for at least two decades
  that parallel machines will eventually displace sequential ones in
  computationally intensive domains.  This, however, has not
  happened.'' --- Leslie
  Valiant~\cite{Valiant_CACM1990}\footnote{Guess when this was
    written?  You may be surprised.}
\end{quote}

For several decades, computer scientists have predicted that the dawn
of the age of parallel computing was ``right around the corner'' and
that sequential processing would soon fade into obsolescence
(consider, for example, the above quote).  Yet, until very recently,
they have been wrong.  The relentless progress of Moore's Law for
several decades has ensured that most of the world's problems could be
solved by single-processor machines, save the needs of a few
(scientists simulating molecular interactions or nuclear reactions,
for example).  Couple that with the inherent challenges of
concurrency, and the result has been that parallel processing and
distributed systems have largely been confined to a small segment of
the market and esoteric upper-level electives in the computer science
curriculum.

However, all of that changed around the middle of the first decade of
this century.  The manner in which the semiconductor industry had been
exploiting Moore's Law simply ran out of opportunities for
improvement:\ faster clocks, deeper pipelines, superscalar
architectures, and other tricks of the trade reached a point of
diminishing returns that did not justify continued investment.  This
marked the beginning of an entirely new strategy and the dawn of the
multi-core era~\cite{Olukotun_Hammond_2005}.  Unfortunately, this
radical shift in hardware architecture was not matched at that time by
corresponding advances in how software could be easily designed for
these new processors (but not for lack of trying~\cite{McCool_2008}).
Nevertheless, parallel processing became an important issue at the
forefront of everyone's mind---it represented the only way forward.

At around the same time, we witnessed the growth of large-data
problems.  In the late 1990s and even during the beginning of the
first decade of this century, relatively few organizations had
data-intensive processing needs that required large clusters:\ a
handful of internet companies and perhaps a few dozen large
corporations.  But then, everything changed.  Through a combination of
many different factors (falling prices of disks, rise of
user-generated web content, etc.), large-data problems began popping
up everywhere.  Data-intensive processing needs became widespread,
which drove innovations in distributed computing such as
MapReduce---first by Google, and then by Yahoo and the open source
community.  This in turn created more demand:\ when organizations
learned about the availability of effective data analysis tools for
large datasets, they began instrumenting various business processes to
gather even more data---driven by the belief that more data leads to
deeper insights and greater competitive advantages.  Today, not only
are large-data problems ubiquitous, but technological solutions for
addressing them are widely accessible.  Anyone can download the open
source Hadoop implementation of MapReduce, pay a modest fee to rent a
cluster from a utility cloud provider, and be happily processing
terabytes upon terabytes of data within the week.  Finally, the
computer scientists are right---the age of parallel computing has
begun, both in terms of multiple cores in a chip and multiple machines
in a cluster (each of which often has multiple cores).

Why is MapReduce important?  In practical terms, it provides a very
effective tool for tackling large-data problems.  But beyond that,
MapReduce is important in how it has changed the way we organize
computations at a massive scale.  MapReduce represents the first {\it
  widely-adopted} step away from the von Neumann model that has served
as the foundation of computer science over the last half plus century.
Valiant called this a {\it bridging model}~\cite{Valiant_CACM1990}, a
conceptual bridge between the physical implementation of a machine and
the software that is to be executed on that machine.  Until recently,
the von Neumann model has served us well:\ Hardware designers focused
on efficient implementations of the von Neumann model and didn't have
to think much about the actual software that would run on the
machines.  Similarly, the software industry developed software
targeted at the model without worrying about the hardware details.
The result was extraordinary growth:\ chip designers churned out
successive generations of increasingly powerful processors, and
software engineers were able to develop applications in high-level
languages that exploited those processors.

Today, however, the von Neumann model isn't sufficient anymore:\ we
can't treat a multi-core processor or a large cluster as an
agglomeration of many von Neumann machine instances communicating over
some interconnect.  Such a view places too much burden on the software
developer to effectively take advantage of available computational
resources---it simply is the wrong level of abstraction.  MapReduce
can be viewed as the first breakthrough in the quest for new
abstractions that allow us to organize computations, not over
individual machines, but over entire clusters.  As Barroso puts it,
the datacenter {\it is} the
computer~\cite{Barroso_Holzle_2009,Patterson_CACM2008}.

To be fair, MapReduce is certainly not the first model of parallel
computation that has been proposed.  The most prevalent model in
theoretical computer science, which dates back several decades, is the
PRAM~\cite{JaJa_1992,Grama_etal_2003}.\footnote{More than a
  theoretical model, the PRAM has been recently prototyped in
  hardware~\cite{Wen_Vishkin_2008}.} In the model, an arbitrary number
of processors, sharing an unboundedly large memory, operate
synchronously on a shared input to produce some output.  Other models
include LogP~\cite{Culler_etal_1993} and BSP~\cite{Valiant_CACM1990}.
For reasons that are beyond the scope of this book, none of these
previous models have enjoyed the success that MapReduce has in terms
of adoption and in terms of impact on the daily lives of millions of
users.\footnote{Nevertheless, it is important to understand the
  relationship between MapReduce and existing models so that we can
  bring to bear accumulated knowledge about parallel algorithms; for
  example, Karloff et al.~\cite{Karloff_etal_2010} demonstrated that a
  large class of PRAM algorithms can be efficiently simulated via
  MapReduce.}

MapReduce is the most successful abstraction over large-scale
computational resources we have seen to date.  However, as anyone who
has taken an introductory computer science course knows, abstractions
manage complexity by hiding details and presenting well-defined
behaviors to users of those abstractions.  They, inevitably, are
imperfect---making certain tasks easier but others more difficult, and
sometimes, impossible (in the case where the detail suppressed by the
abstraction is exactly what the user cares about).  This critique
applies to MapReduce:\ it makes certain large-data problems easier,
but suffers from limitations as well.  This means that MapReduce is
not the final word, but rather the first in a new class of programming
models that will allow us to more effectively organize computations at
a massive scale.

So if MapReduce is only the beginning, what's next beyond MapReduce?
We're getting ahead of ourselves, as we can't meaningfully answer this
question before thoroughly understanding what MapReduce can and cannot
do well.  This is exactly the purpose of this book:\ let us now begin
our exploration.

\section{What This Book Is Not}

Actually, not quite yet\ldots A final word before we get started.
This book is about MapReduce algorithm design, particularly for text
processing (and related) applications.  Although our presentation most
closely follows the Hadoop open-source implementation of MapReduce,
this book is explicitly {\it not} about Hadoop programming.  We don't
for example, discuss APIs, command-line invocations for running jobs,
etc.  For those aspects, we refer the reader to Tom White's excellent
book, ``Hadoop:\ The Definitive Guide'', published by
O'Reilly~\cite{White_2009}.
